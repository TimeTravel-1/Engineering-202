\documentclass[11pt]{article}

%%%%%%%%%%%%%% LATEX SAMPLE FILE %%%%%%%%%%%%%%%%
% A line which starts with a % sign
% is called a COMMENT. It is IGNORED
% by the LaTeX processor.

% Include math
\usepackage{amsmath,amsthm,amssymb}
% Include links
\usepackage{hyperref}


%%%%%%%%%%%%%  THEOREMS  %%%%%%%%%%%%%%%%%
% Let's define some theorem environments
% To use later in the paper
\theoremstyle{plain} % other options: definition, remark
\newtheorem*{theorem}{Theorem}
\newtheorem*{lemma}{Lemma}
% By including [theorem], the lemma follows the numbering of theorem
% e.g. Thm 1, Lemma 2, Thm 3, Thm 4, \dots
\theoremstyle{definition}
\newtheorem*{definition}{Definition} % the star prevents numbering

\theoremstyle{example}
\newtheorem*{example}{Example}
% Remarks
\theoremstyle{remark}
\newtheorem*{remark}{Remark}

\DeclareMathOperator{\sinc}{sinc}


%%%%%%%%%%%%%%  PAGE SETUP %%%%%%%%%%%%%%%%%
% LaTeX has big default margins
% The following sets them to 1in
\usepackage[margin=1.5in]{geometry}

% The following sets up some headers
\usepackage{fancyhdr}
\pagestyle{fancy}
\lhead{Engineering Statics} % Left Header
\rhead{\thepage} % Right Header
\cfoot{} % Center Foot (empty)






%%%%%%%%%%%%% SHORTCUTS %%%%%%%%%%%%%%%%%%%%
% You can define your own shortcuts too.
% Examples of custom commands
\newcommand{\half}{\frac{1}{2}}
\newcommand{\cbrt}[1]{\sqrt[3]{#1}}

\begin{document}

% Set up a title
\title{ENGG 202}
\author{David Ng}
\date{Winter 2017}
\maketitle

% This line makes a ToC
\tableofcontents

% This line starts a new page
\eject

%%%%%%%%%%%%% January 11 %%%%%%%%%%%%%%%%%%%%

\section{January 9, 2016}
\subsection{Vector Algebra Review}

A \textbf{vector} is a directed segment with two basic operation, \textbf{vector sum} and \textbf{scalar multiplication}. To compute the vector sum, we take the diagonal of the parallelogram formed by the two vectors. To compute the scalar multiplication, we take a collinear vector with its magnitude multiplied by the scalar constant. We can denote a coordinate system with $\vec{i}$, $\vec{j}$, and $\vec{k}$ as unit vectors each orthogonal to each other. When we define a vector $\vec{v}$ in this coordinate system, we can uniquely express it as a linear combination of the three unit vectors. Note that this is equivalent to computing the algebraic sum of the vector's corresponding components. The \textbf{dot product} is a real number which is defined as 
$$\vec{u} \cdot \vec{v} = \|u\|\|v\|\cos\alpha.$$
We note the following properties that arise
$$\vec{u} \cdot \vec{v} = u_iv_i + u_jv_j + u_kv_k,$$
$$\vec{u} \cdot \vec{u} = \|u\|^2.$$
\begin{example}
Let $\vec{v_1}$ be a vector with a length of $27$ units and an angle of $30^{\circ}$ and let $\vec{v_2} $be a vector with a length of $21$ and a slope of $4/3$.
\end{example}
We can first express the vectors as a sum of their components. We can then sum the individual components to obtain $\vec{v_1} + \vec{v_2} = 10.783\vec{i} + 30.300\vec{j}$.

\begin{remark}We shall label the angle between the coordinate axes as $\alpha$, $\beta$, and $\gamma$ respectively. These angles always range from $0^{\circ} \leq \alpha, \beta, \gamma \leq 180^{\circ}$. In a vector space, a vector may have $0$ length when units do not allow for an interpretation of length. 
\end{remark}

\section{January 11, 2016}


\subsection{Newton's Laws of Motion}

Let us define a \textbf{particle} as a body of negligible dimensions. That is, we consider a particle to be a body whose dimensions are considered to be near zero, so we may analyze it as a mass concentrated at a point. We now state Newton's laws:

\begin{enumerate}
	\item \textbf{Law I} (Inertia): A particle remains at rest or continues to move with uniform velocity (in a straight line with a constant speed) if there is no unbalanced force acting on it. 
	\item \textbf{Law II}: The acceleration of a particle is proportional to the vector sum of forces acting on it, and is in the direction of this vector sum. If there is a force acting on a particle, then the particle either moves non-uniformly along a straight line, or along a curved line. That is, its motion is governed by $\vec{F} = m\vec{a}$.
	\item \textbf{Law III}: The forces of action and reaction between interacting bodies are equal in magnitude, opposite in direction, and collinear. 
\end{enumerate}

\section{January 13, 2016}
\subsection{Forces}

A \textbf{force} is defined as an action of one body on another. Since it is a vector quantity, we must specify force with a magnitude, direction, and point of application. In this course, we refer to systems as free body diagrams. This means that we are isolating a system that describes the external forces, as opposed to the internal forces. We also postulate that the effect of several forces on a point is the effects of the sum of the forces on the point. When dealing with mechanics of rigid bodies, we concern ourselves with only the net external effects of external forces. In this case, it is not necessary to restrict the action to a single point only. \textbf{The Principle of Transmissbility} states that a force may be applied at any point on its given line of action without altering the resultant effects of the force external to the rigid body on which it acts.

\begin{example}
Let point $A = (-3, 7)$ and point $B = (10, -8)$, and let $\|\vec{F}\| = 125 N$  along this line from $A$ to $B$. Determine $F_x$ and $F_y$.
\end{example}

We first find the vector $\overrightarrow{AB}$. This is the vector, $13 \vec{i} -15\vec{j}$. We then find the unit vector in this direction, and multiply component-wise with the force to determine $F_x$ and $F_y$ Since the unit vector is $0.655\hat{i} -0.756\hat{j}$, multiplying by $125N$ gives the force as $\left(81.9\hat{i} -94.5\hat{j}\right) N$. 


\begin{example}
Let $F_1 = 1400$ $N$ vertically upwards, and $F_2 = 800$ $N$ in an unspecified angle from the first. Determine this angle so that the resultant force is $2000$ $N$. 
\end{example}

We note that vertically, the component is $1400$ $N$. The vertical component of $F_2$ is $800\sin(90-\theta)$ $N$, and the horizontal component of $F_2 $ is $800\cos(90-\theta)$. Separating the resultant vector into components, we note that the magnitude of the resultant force $$F = \sqrt{\left(1400+800\sin(90-\theta)\right)^2 + \left(800\cos(90-\theta)\right)^2}.$$
Setting $F = 2000$ $N$ and solving for $\theta$ gives $\theta \approx  35.51460955753.$

\section{January 16, 2016}
\subsection{Cross Product}

The \textbf{cross product} between vectors $\vec{a}$ and $\vec{b}$ is denoted by $\vec{a}\times\vec{b}$, and the result between the cross product of two vectors is another vector. We note that two non-collinear vectors always reside upon a unique plane. Geometrically, the cross product forms a vector which is perpendicular to this plane. The magnitude of $\|\vec{u} \times\vec{v}\| = \|u\| \|v\| \sin\theta$. Furthermore, we use the right hand rule to determine the direction that the resultant vector points. We note that while the cross product is not commutative, it is anti-commutative (and skew symmetric). We recall that we can determine the result of the cross product by evaluating the determinant of the $3\times3$ matrix with the unit vectors in the first row, and the two vectors in the second and third rows. 

\subsection{Moment}

The rotational tendency of a force is known as the moment $\vec{M}$ of a force. It is often commonly referred to as torque. The moment of a force $\vec{F}$ about a point $A$ may be represented by the cross product
$$\vec{M} = \vec{r}\vec{F}$$
where $r$ is a position vector which extends from point $A$ to any point on the line of action of $\vec{F}$. 

\begin{theorem}[Varignon's Theorem]
The moment of a force about any point is equal to the sum of the moments of the components of the force about the same point. Consider the force $\vec{R}$ acting in the plane. The forces $\vec{P}$ and $\vec{Q}$ represent the x component and y component respectively of $\vec{R}$. The moment of $\vec{R}$ about point $O$ is 
$$\vec{M_O} = \vec{r}\times \vec{R},$$
where $r$ is the vector from point $O$ to an arbitrary point along the force $\vec{R}$. Since $\vec{R} = \vec{P} + \vec{Q}$, we may write 
$$\vec{r} \times \vec{R} = \vec{r} \times(\vec{P}+\vec{Q}).$$
Applying the distributive law for cross products, we obtain,
$$\vec{M_O} = \vec{r}\times \vec{R} =\vec{ r} \times \vec{P} + \vec{r} \times \vec{Q}.$$

\end{theorem}


\section{January 18, 2016}
\subsection{Moment Examples}

To solve, we first choose a coordinate system. Then, we tabulate the coordinates of salient points. 

\begin{example}
Let there be points $A$, $B$, and $C$, where $A$ is a point on the bottom, $B$ is directly $1.6m$ above $A$, and $C$ is located $1.6m$ away from $B$ in a northwest direction of $45^{\circ}$. Determine the moment with respect to point $A$ when a force of $30N$ is applied at point $C$ at $45^{\circ}$ to the x-axis. 
\end{example}

We apply the traditional xy coordinate system, then tabulate the $x$, $y$, and $z$ components of points $A$, $B$, and $C$. The x component of $A$ and $B$ is $1.6\cos\left(45^{\circ}\right) = 1.131m$. The y component of $B$ is $1.6m$, and the y component of $C$ is $1.6m+1.6\sin\left(45^{\circ}\right)m = 2.731m$. We note that by splitting the force into xy components at $45^{\circ}$, we get $$\vec{F} = 21.21N\vec{i} + 21.21N\vec{j}.$$
To find the moment we recall that $\vec{M}_A =\vec{r}_{AC}\times\vec{F}$. Thus, solving this, we obtain
\begin{align*}
\vec{M}_A &= \left(\vec{r}_C-\vec{r}_A\right)\times\vec{F} \\
&= \left[(-1.131)\vec{i} + (2.731)\vec{j} + 0 \vec{k}\right] \times \left[(21.21\vec{i} + 21.21\vec{j}+0\vec{k}\right]\\
&= -81.91Nm\vec{k}
\end{align*}

\begin{example}
Suppose we are given two concentric circles, of 5 inches and 8 inches respectively. Point $C$ is the center, point $P$ is on the ground, and point $B$ is a point on the inner circle which is extended by a tangential string at point $T$ with $32 lb$. Determine the moment with respect to point $C$. Secondly, determine what $\theta$ must be so that $\vec{M}_P = 0$. 
\end{example}

We note that $\vec{M}_C = -(32lb)(5") = -160lb\cdot in$. To solve the second question, we node that $T$, $B$, and $P$ lie on a line. Then, $\cos\theta = \frac{5}{8}$. Solving for $\theta$ gives an angle of $-51.32^{\circ}$. 

\begin{example}[2.49]
Let the mass centered at point $A$ of a ball on a hand be $8lb$. The forearm is $5lb$, centered at location $G$, which is extended horizontally for $13''$ and vertically downwards by $6''$ at an angle of $-35^{\circ}$. $T$ is $2''$ above $O$, the point of the elbow. Determine the force that the biceps must produce at $T$ so that $\vec{M} = 0$. 
\end{example}

We note that the moment is the sum of all three forces.Therefore, 
\begin{align*}
	M_0 &= -(8lb)(13'')-(5lb)(6'')\sin(55^{\circ})+F(2'') \\
	0 &= -(8lb)(13'')-(5lb)(6'')\sin(55^{\circ})+F(2'')
\end{align*}
Solving this gives $F = 64.29lb$. 

\section{January 20, 2016}
\subsection{Moment Examples Cont'd}

\begin{example}[2.58]
\end{example}
We solve this question by resolving $\vec{F}$ into components by using Varignon's Theorem. We note that the horizontal and vertical components are respectively $\vec{F}\cos\alpha$ and $\vec{F}\sin\alpha$, with the vertical component directed downwards. At point $A$, it has coordinates relative to $O$ of $(290, 190)$. Let us consider moment $M_0$ as a scalar, with the convention that a clockwise rotation is negative, and a counterclockwise rotation is positive. Then, we have
\begin{align*}
M_0^F &= M_0^{F_x} + M_0^{F_y}\\
&= -(F\sin\alpha)(190mm) -(F\cos\alpha)(290mm)\\
&= -41.54Nm
\end{align*}
To solve the second part of the question, we note that $\alpha$ varies over the real numbers, so we may evaluate the derivative at $0$ to determine the maximum. The derivative of moment with respect to $\alpha$ is
\begin{align*}
\frac{\mathrm d M_0}{\mathrm d \alpha} &= -F\cos\alpha(190mm) +F\sin\alpha(290mm)\\
&= 0\\
\tan\alpha &= \frac{190}{290}\\
&= 0.655\\
\alpha &= 33.23^{\circ} \text{ or } 213.23^{\circ}
\end{align*}
Using this and evaluating $M_0$, we obtain $$M_0 = -41.60Nm.$$

A \textbf{force couple} is any two forces $\vec{F_1}$ and $\vec{F_2}$ equal in magnitude along parallel lines in opposite directions. We can show that the moment of a couple about any point is independent of the point.

\section{January 23, 2016}
\subsection{Summary}

\begin{enumerate}
	\item We can consider forces as vectors. 
	\item The combined effect of several concurrent forces $\vec{F}_1, \vec{F}_2, ...,\vec{F}_n$ is the same as the effect of their vector sum $\vec{R} = \vec{F}_1 + \vec{F}_2 + ...+\vec{F}_n$. 
	\item The principle of transmissibility states that the effect of the force is not altered by a translation along the line of action of the force. 
	\item We defined the moment of a force $\vec{F}$ with respect to a point $P$ by taking a point $Q$ on the line of action of the force, called $Q$. The directed vector from $P$ to $Q$ gives $\vec{r}_{PQ}$, which is used to define the moment $$\vec{M}_P^{\vec{F}} = \vec{r}_{PQ} \times \vec{F}$$ with a counter-clockwise rotation using the right hand rule with the resultant vector pointing outwards. 
\end{enumerate}

\begin{example}
Consider two forces of equal magnitude in opposite directions along parallel lines of action, starting at points $B$ and $C$ respectively. We now consider the moment about point $A$, located in the center. Determine the moment about point $A$
\end{example}

The moment would be 
\begin{align*}
	\vec{M}_A &= \vec{r}_{AB} \times \vec{F} + \vec{r}_{AC} \times \left(-\vec{F}\right) \\
	&=\left(\vec{r}_{AB} - \vec{r}_{AC}\right) \times \vec{F}\\
	&= \vec{r}_{CB} \times \vec{F}
\end{align*}
with a magnitude of $\left|\vec{M}_A\right| = Fd$ where $d$ is the distance from $B$ to $C$. We note that the moment does not depend on point $A$. 

\subsection{Force Couple System}

A \textbf{force-couple system} results from the desire to translate a force outside of its line of action. We can form a couple of this force with a force on the line on which we want to translate with an additional force on this line pointing in the opposite direction to cancel the net effect. The result of the translation is therefore a force along with a couple. We note that any system of forces, not necessarily concurrent, can be reduced to a force couple system. Couples are free vectors in the sense that they reside on a plane of action, and not a line or point of action. 

\begin{example}
Suppose point $A$ is $7.5m$ vertically from point $O$, and there is a force pointing $30^{\circ}$ south from west with $32kN$. point $B$ has coordinates $(6m, 2m)$. Replace $\vec{F}$ with a force-couple system at point $O$, then at point $B$. 
\end{example}
In the case of replacing for point $O$, we have 
\begin{align*}
\vec{M}_O^{\vec{F}} &= F\cos\left(30^{\circ}\right)(7.5m) \\
&= 20.78kNm
\end{align*}
Thus, this moment is combined with the force of $32kN$ in the direction of $30^{\circ}$ south from west at point $O$. For point $B$, we find that 
\begin{align*}
	\vec{M}_B^{\vec{F}} &= F\cos\left(30^{\circ}\right)(7.5m-2m) + F\sin\left(30^{\circ}\right)(6m)\\
	&= 24.84kNm
\end{align*}
with the same force in the same direction, forming a force-couple system. 

\section{January 25, 2016}
\subsection{Reduction of Arbitrary Force System to a Point}
\begin{remark}
Forces are sliding vectors that can be moved along its line of action, and couples are free vectors that can be moved along any parallel line. 
\end{remark}

We note that every force system is equivalent to a single force-couple system. This can be seen first by choosing any arbitrary point $P$. Each force is then moved in succession to point $P$ by converting it into a force-couple system. Since this is a concurrent system of forces, we find that the net effect is equal to the vector sum of the forces. The same can be applied to obtain the net effect of the couples. We are left with a single force and a single couple. We may also selectively choose points so that the couple $\vec{C} = 0$. 

\begin{example}
Let an object with its bottom at point $B$ rising a total of $90''$ have a force of $250lb$ to the left, $650lb$ to the right, and $300lb$ to the left at $30''$, $60''$, and $90''$ from the base $B$ respectively. Replace the forces by a single force.
\end{example}

We reduce the system to point $B$. Adding the forces, we obtain $100lb$ of force to the right. Our couple is therefore $(300lb)(90'')-(650lb)(60'')+(250lb)(30'') = -4500lb*in$. We note that in this case, we find that when we take the point $P$ to be $45''$, then the net couple cancels out. This arises by solving 
$$(300lb)(90''-x) -(650lb)(60''-x) + (250lb)(30''-x)=0$$for $x$. Therefore, if we replace the forces by a single force without any couples, the resulting force would be $100lb$ at $45''$. 



\begin{example}
Suppose we have a truss where a force directed from $30^{\circ}$ west of north of $20kN$, a force directed from the north of $30kN$, and a force directed from the east of $25kN$ is applied to a point located at $(9m, 5m)$. Find the intercepts of the resultant of these three forces. 
\end{example}
To find the $x$ intercept that would result from these forces, we make use of the moment solve for $x$ in the following expression,
$$(25kN)(5m)-(30kN)(9m-x)-(20kN)(9m-x)\cos\left(30^{\circ}\right)-(20kN)(5m)\sin\left(30^{\circ}\right) =0.$$
The expression to find $y$ is synonymous. 

\section{January 27, 2017}

\subsection{Force Couple Systems Cont'd}

\begin{example}
Suppose a force of $10kN$ is applied downwards at $0m$, a force of $4kN$ applied upwards at $4m$, point $B$ at $7m$, and a force of $12kN$ applied downwards at $10m$ on a horizontal beam spanning $10m$. If we know that the resultant passes through $B$ determine $M$. Secondly, find a force-couple system at $0m$ equivalent to the three forces of the couple.  
\end{example}

We note that the force is simply the vector sum of all the forces. Since we have no moment at $B$, the force is $18kN$ downwards. Thus, the sum of moments at $B$ is 
\begin{align*}
\sum M_B &=0\\
 &= (-7m)(10kN) -(3m)(-4kN) + (3m)(12kN) -M\\
 M &= -22kNm
\end{align*}
Secondly, we note that the force at $0m$ would be $18kN$ downwards. Since we want to determine the moment, we use our previous result to obtain 
$$M = (18kN)(7m) = 126kNm.$$

\subsection{Equilibrium}

To achieve \textbf{equilibrium}, we define it such that $\sum \vec{F} = 0$ and $\sum \vec{C} = 0$ are necessary and sufficient conditions to attain equilibrium. We first consider 2D cases before we consider 3D. We can have \textbf{fully fixed support} (as in the case of a clamp), that prevents displacement in both the x and y direction, and also prevents rotation. We can also have \textbf{hinge support} that prevents displacement in both directions, but permits rotation. \textbf{Mounted hinge support} involves a hinge mounted on a slider in either the x or y direction. This prevents only displacement in either the x or y direction. We note that what we prevent (forces or moments) is applied as a reaction to the system to satisfy the requirements of equilibrium. 

\section{January 30, 2016}
\subsection{Equilibrium in Two Dimensions}

A \textbf{rigid body} is one where distances between internal points always remain constant even when forces are applied. In two dimensions, a rigid object has 3 degrees of freedom, $$\sum F_x = 0,$$ $$\sum F_y = 0,$$  $$\sum M_A = 0.$$ To eliminate a degree of freedom, we introduce a support that restricts the movement of the rigid object. We recall the main forms of support are \textbf{fixed support} (built in), \textbf{pin support} (hinge), and roller support. To manifest the reactions of the supports on the object, we use free body diagrams. For instance, to prevent a horizontal force, the support applies a horizontal force. To prevent a vertical force, it applies a vertical force. To prevent a rotation, it applies a couple. We note that a free body diagram must be drawn separately from the diagram depicting the system. 

\begin{enumerate}
	\item \textbf{Fixed support} prevents all three degrees of freedom. Thus, it has 3 reaction forces on its free body diagram, for forces in both directions and the couple for rotation. 
	\item \textbf{Pin support} prevents two degrees of freedom. Thus, it has 2 reaction forces on its free body diagram for forces in both direction. 
	\item \textbf{Roller support} prevents one degree of freedom. Thus, it has 1 reaction force on its free body diagram for a force in one direction. 
\end{enumerate}

\begin{example} 
Suppose we are presented with a fixed hinge at point $A$. A beak extends $8m$, at which point $B$ is fixed to a rocker. At $5.6m$ from $A$, is a mass of $220kg$ attached, while the beam is $450kg$. Find the reactions at $A$ and $B$. 
\end{example}

We note that the rocker is functionally the same as a roller. We then draw a free body diagram noting the effect of gravity on the centre of mass of the beam, and the effect of gravity on the attached mass. At point $A$, we have two reaction forces, one vertical and another horizontal ($A_x$ and $A_y$), and at point $B$, we have one vertical reaction force ($B_y$). We require that the forces and couples must sum to 0. Thus we obtain
$$\sum F_x = 0 = A_x,$$
$$\sum F_y = A_y - W_1 - W_2 +B_y,$$
$$\sum M_A = -W_1(4m) - W_2(5.6m)+B_y(8m).$$
Solving the third equation first for $B_y$, we find that it equals $3718N$. Solving the second equation gives $A_y = 2855N$, and $A_x = 0$. 











\section{February, 1, 2017}
\subsection{Two-Force Members}

Suppose we have two sets of forces acting on points $A$ and $B$. We can resolve this into forces $F_A$ and $F_B$ around the respective points. \textbf{Two-force members} occur without couples. 
We note then that the moments around both points are zero,
$$\sum M_A = 0,$$
$$\sum M_B = 0.$$
Therefore, force $F_A$ passes through point $B$, and $F_B$ passes through point $A$ through their lines of action. 


\begin{example}
Suppose we have an inverted $L$ shape with vertices $A$, $B$, and $C$, where $C$ is located on a fixed hinge. From point $B$, we have a similar inverted $L$ shape extending to the right and downwards to point $D$. Point $D$ is also located on a fixed hinge. A $150N$ force is applied downwards at point $A$ which is located $1m$ to the left and $1m$ above point $C$. Point $D$ is located $1m$ right and $1m$ downwards from point $B$.
\end{example}

We first draw the free body diagrams for both the first and second segment. For the first system, at $A$ is $150N$ downwards, at point $B$ is a force $V_B$ and $H_B$ directed upwards and to the right respectively, and at point $C$ is a force of $V_C$ and $H_C$ directed upwards and to the right. For the second system, we note that the forces at point $B$ cancel those from the first system. The force on $D$ is $V_D$ and $H_D$ directed upwards and to the right. Note that the second system is a two force member, so the line of action crosses $B$ and $D$. The angle is $\arctan(0.5/1) = 26.565^{\circ}$. The force on $D$ which we label $R$ is therefore the same as that on $B$ by the first system, and in the opposite direction as the force on $B$ by the second system. Thus, we find that
$$\sum F_x = H_C + R\cos(\alpha) = 0,$$
$$\sum F_y = V_C -150N + R\sin(\alpha) = 0,$$
$$\sum M_B = (150N)((1m) + H_C(1m) = 0.$$
Solving this system gives $H_C = -150N$, $R = 167.7N$ and $V_C = 225N$, where $R$ is in the direction previously specified. 

\begin{remark}
Independently with no connections the two segments have 3 degrees of freedom each. When we fix them at point $B$, we lose two degrees of freedom. Thus, we have a total of 4 degrees of freedom. The two hinges each prevent 2 degrees of freedom since they only permit rotation. This system is therefore minimally supported. 
\end{remark}

\subsection{Three-Force Members}

Suppose we have three sets of forces located at points $A$, $B$, and $C$. This results in forces $F_A$, $F_B$, and $F_C$ with lines of action that may meet at a single point. 

\begin{example}
Suppose a $75^{\circ}$ incline measured from the left and a $30^{\circ}$ incline measured from the right support a sphere which is placed between them. The sphere has a weight of $W$, and makes contact with the respective inclines at points $A$ and $B$. The reaction at point $B$, since there is no force of friction, is perpendicular to the surface. The force $R_B$ therefore passes through the center. The same argument is applied to force $R_A$ which also passes through the center of the sphere. This forms a three-force member. We can solve this where $\sum F_x = 0$ and $\sum F_y = 0$. 
\end{example}

\section{February 3, 2017}
\subsection{Three-Force Members Cont'd}

\begin{example}
Suppose we have a uniform $L$ shaped bar (with mass distributed proportionally with length). The long end has a length of $2L/3$, and the short end has a length of $L/3$. This is perched against a wall and the floor. The top angle is $30^{\circ}$. Determine the reaction forces at the two points which make contact with the wall and the floor, called points $A$ and $O$ respectively. 
\end{example}

We shall draw a free body diagram. We note that since the object is uniform, we place the weights $W_1$ in the middle of the long length and $W_2$ in the middle of the short length. At $A$, we have a horizontal force, whereas at $O$, we have both a horizontal and vertical force. By comparing the horizontal and vertical force components, we obtain
$$\sum F_x = H_A-H_O=0,$$
$$\sum F_y = -W_1-W_2+V_O = 0,$$
$$\sum M_O = -H_A\frac{2L}{3}\cos\left(30^{\circ}\right)+W_1\frac{L}{3}\sin\left(30^{\circ}\right)-W_2\frac{L}{6}\cos\left(30^{\circ}\right)=0.$$

\subsection{Linear Algebra}
Given a system of equations, we can combine the sum of forces in different directions. This would give us a sum of forces in another direction, different from the original two directions. Combining the equation of a moment with the sum of forces gives the equation of a moment with respect to another point. We note then that there is an inherent connection between linear algebra and the force systems. 











































































\end{document}
